\documentclass[12pt]{article}

% Packages
\usepackage{amsmath}      % For mathematical notation
\usepackage{amsfonts}     % For fonts like \mathbb{R}
\usepackage{amssymb}      % For additional symbols
\usepackage{graphicx}     % For including graphics
\usepackage{algorithm}    % For pseudocode and algorithms
\usepackage{algorithmic}  % For pseudocode formatting
\usepackage{hyperref}     % For hyperlinks (internal and external)

\title{Eigenvalue Computation of Matrices Using [Your Language]}
\author{Your Name}
\date{\today}

\begin{document}

\maketitle

\begin{abstract}
This report presents the development of a program that computes all eigenvalues of a given square matrix. The matrix eigenvalues are critical in many applications across science and engineering. This report discusses the theory of eigenvalue decomposition, the implementation of the algorithm, and presents the results of testing on various matrices.
\end{abstract}

\tableofcontents
\newpage

\section{Introduction}
The computation of eigenvalues is a fundamental problem in numerical linear algebra with wide applications in fields like physics, computer graphics, and machine learning. In this report, we discuss an implementation of an algorithm to compute all eigenvalues of a matrix in [C/Python/Rust]. The choice of algorithm and its implementation is guided by the need for computational efficiency and numerical stability.

\section{Theory}
\subsection{Eigenvalues and Eigenvectors}
For a given square matrix \(A \in \mathbb{R}^{n \times n}\), a scalar \(\lambda\) is called an eigenvalue if there exists a non-zero vector \(v \in \mathbb{R}^n\) such that:
\[
A v = \lambda v
\]
The vector \(v\) is called the eigenvector corresponding to the eigenvalue \(\lambda\). Eigenvalues have significant applications in stability analysis, vibration modes, and PCA (Principal Component Analysis).

\subsection{Algorithm for Eigenvalue Computation}
Several numerical methods can be used to compute eigenvalues, including:
\begin{itemize}
    \item \textbf{Power Iteration}: A simple method for finding the largest eigenvalue.
    \item \textbf{QR Algorithm}: A more general and reliable method for computing all eigenvalues.
    \item \textbf{Jacobi Method}: Suitable for symmetric matrices.
\end{itemize}

The implementation discussed in this report uses the [QR Algorithm/other algorithm] for robustness and accuracy.

\section{Methodology}
\subsection{Language Choice: [C/Python/Rust]}
For this implementation, the choice of language was based on the trade-off between performance and ease of development. Python provides excellent libraries (like NumPy and SciPy) for linear algebra, while C is known for its speed, and Rust offers a modern, safe alternative with performance akin to C.

\subsection{Implementation Details}
The algorithm follows these steps:
\begin{enumerate}
    \item Input a matrix \( A \).
    \item Apply the QR algorithm to decompose the matrix into \( A = Q R \), where \( Q \) is an orthogonal matrix and \( R \) is upper triangular.
    \item Update the matrix \( A \) as \( A \gets R Q \).
    \item Repeat until \( A \) converges (i.e., it becomes diagonal).
    \item The diagonal elements of the resulting matrix are the eigenvalues.
\end{enumerate}

The program is designed to handle both real and complex matrices. It is optimized for efficiency, with careful attention to numerical stability.

\section{Code Implementation}
\subsection{Python Example}
Here is a simplified version of the Python code used for the eigenvalue computation:

\begin{verbatim}
import numpy as np

def qr_algorithm(A):
    n = A.shape[0]
    for i in range(100):  # Max iterations
        Q, R = np.linalg.qr(A)
        A = np.dot(R, Q)
    return np.diagonal(A)

# Example usage
A = np.array([[4, -2], [-2, 3]])
eigenvalues = qr_algorithm(A)
print("Eigenvalues:", eigenvalues)
\end{verbatim}

\subsection{C Code Example}
[If you implement in C, show a similar code block.]

\subsection{Rust Code Example}
[If you implement in Rust, provide a similar code block.]

\section{Results and Discussion}
We tested the program on various matrices, including diagonal, symmetric, and non-symmetric matrices. Below are the results for a sample 2x2 matrix:

\[
A = \begin{pmatrix}
4 & -2 \\
-2 & 3
\end{pmatrix}
\]
The computed eigenvalues are [list the results here].

The program successfully computes the eigenvalues with high accuracy and performs efficiently even for larger matrices. Performance benchmarks were conducted to compare execution time against other libraries like NumPy and LAPACK.

\section{Conclusion}
This report describes the implementation of an algorithm to compute the eigenvalues of matrices. The QR algorithm was implemented in [C/Python/Rust] and shown to be effective in calculating all eigenvalues with good performance. Future work includes optimizing the code for even larger matrices and exploring other numerical methods for eigenvalue decomposition.

\section{References}
\begin{itemize}
    \item Higham, N. J. \textit{Accuracy and Stability of Numerical Algorithms}. SIAM, 2002.
    \item Golub, G. H.,$\And$Van Loan, C. F. \textit{Matrix Computations}. Johns Hopkins University Press, 2013.
\end{itemize}

\end{document}
